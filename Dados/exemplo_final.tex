% Options for packages loaded elsewhere
\PassOptionsToPackage{unicode}{hyperref}
\PassOptionsToPackage{hyphens}{url}
%
\documentclass[
]{article}
\usepackage{amsmath,amssymb}
\usepackage{iftex}
\ifPDFTeX
  \usepackage[T1]{fontenc}
  \usepackage[utf8]{inputenc}
  \usepackage{textcomp} % provide euro and other symbols
\else % if luatex or xetex
  \usepackage{unicode-math} % this also loads fontspec
  \defaultfontfeatures{Scale=MatchLowercase}
  \defaultfontfeatures[\rmfamily]{Ligatures=TeX,Scale=1}
\fi
\usepackage{lmodern}
\ifPDFTeX\else
  % xetex/luatex font selection
\fi
% Use upquote if available, for straight quotes in verbatim environments
\IfFileExists{upquote.sty}{\usepackage{upquote}}{}
\IfFileExists{microtype.sty}{% use microtype if available
  \usepackage[]{microtype}
  \UseMicrotypeSet[protrusion]{basicmath} % disable protrusion for tt fonts
}{}
\makeatletter
\@ifundefined{KOMAClassName}{% if non-KOMA class
  \IfFileExists{parskip.sty}{%
    \usepackage{parskip}
  }{% else
    \setlength{\parindent}{0pt}
    \setlength{\parskip}{6pt plus 2pt minus 1pt}}
}{% if KOMA class
  \KOMAoptions{parskip=half}}
\makeatother
\usepackage{xcolor}
\usepackage[margin=1in]{geometry}
\usepackage{graphicx}
\makeatletter
\def\maxwidth{\ifdim\Gin@nat@width>\linewidth\linewidth\else\Gin@nat@width\fi}
\def\maxheight{\ifdim\Gin@nat@height>\textheight\textheight\else\Gin@nat@height\fi}
\makeatother
% Scale images if necessary, so that they will not overflow the page
% margins by default, and it is still possible to overwrite the defaults
% using explicit options in \includegraphics[width, height, ...]{}
\setkeys{Gin}{width=\maxwidth,height=\maxheight,keepaspectratio}
% Set default figure placement to htbp
\makeatletter
\def\fps@figure{htbp}
\makeatother
\setlength{\emergencystretch}{3em} % prevent overfull lines
\providecommand{\tightlist}{%
  \setlength{\itemsep}{0pt}\setlength{\parskip}{0pt}}
\setcounter{secnumdepth}{-\maxdimen} % remove section numbering
\usepackage{booktabs}
\usepackage{longtable}
\usepackage{array}
\usepackage{multirow}
\usepackage{wrapfig}
\usepackage{float}
\usepackage{colortbl}
\usepackage{pdflscape}
\usepackage{tabu}
\usepackage{threeparttable}
\usepackage{threeparttablex}
\usepackage[normalem]{ulem}
\usepackage{makecell}
\usepackage{xcolor}
\ifLuaTeX
  \usepackage{selnolig}  % disable illegal ligatures
\fi
\usepackage{bookmark}
\IfFileExists{xurl.sty}{\usepackage{xurl}}{} % add URL line breaks if available
\urlstyle{same}
\hypersetup{
  pdftitle={Relatório Informativo sobre Dengue},
  pdfauthor={Vigilância Epidemiológica do Estado de Rosas},
  hidelinks,
  pdfcreator={LaTeX via pandoc}}

\title{Relatório Informativo sobre Dengue}
\author{Vigilância Epidemiológica do Estado de Rosas}
\date{2022-07-11}

\begin{document}
\maketitle

\subsection{Sobre}\label{sobre}

O \textbf{Departamento de Vigilância Epidemiológica do Estado de Rosas},
por meio deste boletim informativo\textsuperscript{1}, apresenta
informações gerais sobre a dengue\textsuperscript{2}, assim como uma
breve análise dos dados históricos relativos à situação epidemiológica
da dengue no Estado de Rosas. Entre 2007 e 2012, o município registrou
12781 casos confirmados de dengue e 570 óbitos. A distribuição dos casos
confirmados por semana epidemiológica é apresentada na \emph{Figura 1}.
O número de casos por classificação final são apresentados na
\emph{Tabela 1}.

~

~

\includegraphics[width=1\textwidth,height=\textheight]{Imagens/dengue.png}

\textsuperscript{1} Este relatório foi produzido utilizando a linguagem
\texttt{RMarkdown}.

\textsuperscript{2} As informações sobre esta doença foram baseadas em
conteúdo disponibilizado pelo Ministério da Saúde. Para obter mais
informações, acesse
\href{https://www.gov.br/saude/pt-br/assuntos/saude-de-a-a-z/d/dengue}{este
link}.

\newpage

\subsection{Introdução}\label{introduuxe7uxe3o}

\begin{enumerate}
\def\labelenumi{\arabic{enumi}.}
\tightlist
\item
  O que é Dengue
\end{enumerate}

A dengue é a arbovirose urbana mais prevalente nas Américas,
principalmente no Brasil. É uma doença febril que tem se mostrado de
grande importância em saúde pública nos últimos anos. O vírus dengue
(DENV) é um arbovírus transmitido pela picada da fêmea do mosquito
\emph{Aedes aegypti} e possui quatro sorotipos diferentes (DENV-1,
DENV-2, DENV-3 e DENV-4).

\begin{enumerate}
\def\labelenumi{\roman{enumi}.}
\tightlist
\item
  Principais Sintomas
\end{enumerate}

\begin{itemize}
\tightlist
\item
  Febre alta \textgreater{} 38°C.
\item
  Dor no corpo e articulações
\item
  Dor atrás dos olhos.
\item
  Mal estar.
\item
  Falta de apetite.
\item
  Dor de cabeça.
\item
  Manchas vermelhas no corpo.
\end{itemize}

\begin{enumerate}
\def\labelenumi{\roman{enumi}.}
\setcounter{enumi}{1}
\tightlist
\item
  Transmissão
\end{enumerate}

O vírus da dengue (DENV) pode ser transmitido ao homem principalmente
por via vetorial, pela picada de fêmeas de \emph{Aedes aegypti}
infectadas, no ciclo urbano humano--vetor--humano. Os relatos de
transmissão por via vertical (de mãe para filho durante a gestação) e
transfusional são raros.

\begin{enumerate}
\def\labelenumi{\roman{enumi}.}
\setcounter{enumi}{2}
\tightlist
\item
  Diagnóstico
\end{enumerate}

\begin{itemize}
\item
  Métodos diretos

  \begin{itemize}
  \item
    Pesquisa de vírus (isolamento viral por inoculação em células);
  \item
    Pesquisa de genoma do vírus da dengue por transcrição reversa
    seguida de reação em cadeia da polimerase (RT-PCR);
  \end{itemize}
\item
  Métodos indiretos

  \begin{itemize}
  \item
    Pesquisa de anticorpos IgM por testes sorológicos (ensaio
    imunoenzimático -- ELISA)
  \item
    Teste de neutralização por redução de placas (PRNT);
  \item
    Inibição da hemaglutinação (IH);
  \item
    Pesquisa de antígeno NS1 (ensaio imunoenzimático -- ELISA);
  \item
    Patologia: estudo anatomopatológico seguido de pesquisa de antígenos
    virais por imuno-histoquímica (IHQ).
  \end{itemize}
\end{itemize}

\newpage

\subsection{Análises}\label{anuxe1lises}

\subsubsection{1. Distribuição de casos por semana
epidemiológica}\label{distribuiuxe7uxe3o-de-casos-por-semana-epidemioluxf3gica}

\includegraphics{exemplo_final_files/figure-latex/unnamed-chunk-1-1.pdf}

\newpage

\subsubsection{2. Número de casos por classificação
final}\label{nuxfamero-de-casos-por-classificauxe7uxe3o-final}

\begin{longtable}[t]{rrrrr}
\toprule
Ano & Cura & Óbito & Outro & Ignorado\\
\midrule
1993 & 0 & 0 & 1 & 0\\
2001 & 0 & 0 & 1 & 0\\
2003 & 1 & 0 & 0 & 0\\
2005 & 1 & 0 & 0 & 0\\
2007 & 347 & 27 & 1074 & 50\\
\addlinespace
2008 & 1164 & 453 & 4751 & 219\\
2009 & 29 & 7 & 166 & 8\\
2010 & 86 & 30 & 105 & 13\\
2011 & 3584 & 52 & 541 & 45\\
2012 & 18 & 1 & 2 & 5\\
\bottomrule
\end{longtable}

\end{document}
